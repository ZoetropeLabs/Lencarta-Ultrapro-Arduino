\hypertarget{pingpair_maple_8pde-example}{\section{pingpair\+\_\+maple.\+pde}
}
This is an example of how to use the \hyperlink{class_r_f24}{R\+F24} class on the Maple. For a more detailed explanation, see my blog post\+: \href{http://maniacbug.wordpress.com/2011/12/14/nrf24l01-running-on-maple-3/}{\tt n\+R\+F24\+L01+ Running on Maple}

It will communicate well to an Arduino-\/based unit as well, so it's not for only Maple-\/to-\/\+Maple communication.

Write this sketch to two different nodes, connect the role\+\_\+pin to ground on one. The ping node sends the current time to the pong node, which responds by sending the value back. The ping node can then see how long the whole cycle took.


\begin{DoxyCodeInclude}
\end{DoxyCodeInclude}
 