\hypertarget{_getting_started_8ino-example}{\section{Getting\+Started.\+ino}
}
{\bfseries Updated\+: T\+M\+Rh20 2014 }~\newline


This is an example of how to use the \hyperlink{class_r_f24}{R\+F24} class to communicate on a basic level. Write this sketch to two different nodes. Put one of the nodes into 'transmit' mode by connecting with the serial monitor and ~\newline
 sending a 'T'. The ping node sends the current time to the pong node, which responds by sending the value back. The ping node can then see how long the whole cycle took. ~\newline
 \begin{DoxyNote}{Note}
For a more efficient call-\/response scenario see the Getting\+Started\+\_\+\+Call\+Response.\+ino example. 

When switching between sketches, the radio may need to be powered down to clear settings that are not \char`\"{}un-\/set\char`\"{} otherwise
\end{DoxyNote}

\begin{DoxyCodeInclude}
\end{DoxyCodeInclude}
 